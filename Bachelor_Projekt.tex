%% Pr�ambel
%% Initialisierung
\documentclass[enabledeprecatedfontcommands, a4paper]{scrreprt}    % articel: scrartcl   % book: scrreprt
\usepackage[T1]{fontenc}
\usepackage[latin1]{inputenc}
\usepackage{fancyheadings}

\year=2020
\month=10
\day=13

% Abk�rzungsverzeichnis
\usepackage[withpage, smaller]{acronym}

%% Sprache
\usepackage[ngerman]{babel}

%% Boxstyle
\usepackage{fancybox}

%% Fu�noten
\usepackage{endnotes}

% Querformat paket
\usepackage{pdflscape}

% Einbinden von PDF's
\usepackage{pdfpages}

% Lesezeichenpacket f�r PDF
\usepackage{hyperref}

%% American Mathematical Society Pakete
\usepackage{amsmath,esint}
\usepackage{amsfonts}
\usepackage{amssymb}
\usepackage{amstext}
\usepackage{mathrsfs}
\usepackage{bbm}
\usepackage{cancel}
%\usepackage{subfigure}
\DeclareMathAlphabet\mathbfcal{OMS}{cmsy}{b}{n}
\usepackage{trfsigns}

%% Algorithmen (Pseudocode)
\usepackage[]{algorithm2e}

%% R�mische Zahlen:  z.B. \RM{2}
\newcommand{\RM}[1]{\MakeUppercase{\romannumeral #1{}}}   

%% Textverarbeitung Paket
\usepackage{color}

%% Griechische Buchstaben aufrecht
\usepackage{upgreek}

%% Grafik Paket
\usepackage{graphicx}
\usepackage{float}

%% Include Diagramm Package
\input xy 
\xyoption{all}

%% Include PGFPlots
\usepackage{pgfplots}
\pgfplotsset{compat=1.5}

%% Subfigures
\usepackage{subcaption}

%% Matlab-Files package   %% \lstinputlisting{dateiname.m}
\usepackage{listings}
\lstset{
	language = Matlab,
	showstringspaces = false,
	numbers = left,
	numberstyle = \tiny,
	breaklines = true,
	basicstyle = \tiny,
}

%% Zeileneinr�cken verhindern
 \setlength{\parindent}{0em} 

%% Style
\pagestyle{fancy}
\lhead[\bfseries \title \protect]{\bfseries \title \protect}
\cfoot{Seite: \thepage}

% fuer Zitate und Literaturverzeichnis
\usepackage{natbib}
\bibliographystyle{alpha}



\usepackage{tikz} 
\usetikzlibrary{arrows,decorations.pathmorphing,backgrounds,fit,positioning,shapes.symbols,chains}


%%%%%%%%%%%%%%%%%%%%%%%%%%%%%%%%%%%%%%%%%%%%%%%%%%%%%%%%%%%%%%%%%%%
\begin{document}
\begin{titlepage}
{\centering
\includegraphics[width=0.6\textwidth]{HS_Logo2}\par
\vspace{1cm}
{\scshape\LARGE University of Applied Sciences Trier \par}
\vspace{1.5cm}
{\scshape\Large Bachelor Projektarbeit \par}
\vspace{1.5cm}
{\huge\bfseries Entwicklung und Implementierung eines Can-Interfaces zur pr�zisen
	Lageregelung eines Portalkrans \par}
\vspace{1.5cm}
{\scshape \large BACHELOR OF ENGINEERING \par}
\vspace{0.8cm}
{\scshape\large  Elektrotechnik\par}
{\scshape\large Informationstechnologie und Elektronik  \par}
\vspace{1.0cm}
{verfasst von: \\  \par}
%\vspace{0.2cm}
{\large  Torsten  \textsc{Zimmermann}. [966352]\\  \par}
 \vspace{0.5cm}
 {Teamprojekt mit: \\  \par}
 {\large  Steave  \textsc{Dahm}. [969471]\\  \par}
 \vspace{0.5cm}
betreut von: \par
{\large Prof. Dr. Ing. Matthias \textsc{Scherer} \\  Andreas \textsc{Reis}, M. Sc. \par}
\vspace{0.5cm}
{Abgabedatum: \\  \par}
%\vspace{0.2cm}
{\large \today \par}} %\today
\end{titlepage}

\begin{titlepage}
{\centering
\includegraphics[width=0.6\textwidth]{HS_Logo2}\par
\vspace{1cm}
{\scshape\LARGE University of Applied Sciences Trier \par}
\vspace{1.5cm}
{\scshape\Large Master Thesis \par}
\vspace{1.5cm}
{\huge\bfseries N-Body Langevin-Dynamic-Simulation of magnetic Nanoparticles with MATLAB \par}
\vspace{1.5cm}
{\scshape \large Master of Science \par}
\vspace{0.8cm}
{\scshape\large  Electrical Engineering \par}
{\scshape\large Information Technology and Electronics  \par}
\vspace{1.5cm}
{author: \\  \par}
%\vspace{0.2cm}
{\large  Michael P.  \textsc{Adams}, B. Eng. [957815]\\  \par}
 \vspace{0.5cm}
supervisors: \par
{\large Prof. Dr. Ing. Hellmut \textsc{Hupe} \\  J�rg \textsc{Fusenig}, M. Sc. \par}
\vspace{0.5cm}
{submission date: \\  \par}
%\vspace{0.2cm}
{29. February 2020 \par}} %\today
\end{titlepage}


%%%%%%%%%%%%%%%%%%%%%%%%%%%%%%%%%%%%%%%%%%%%%%%%%%%%%%%%%%%%%%%%%%%


% Hier Startet das Dokument %%%%%%%%%%%%%%%%%%%%%%%%%%%%%%%%%%%%%%%%%%%%%%%%%%%%


{\huge \textbf{Eidesstattliche Erkl�rung}} \\ \\
Ich versichere, die Master-Thesis selbstst�ndig und lediglich unter Benutzung der angegebenen Quellen und Hilfsmittel verfasst zu haben. \\ \\

Ich erkl�re weiterhin, dass die vorliegende Arbeit noch nicht im Rahmen eines anderen Pr�fungsverfahrens eingereicht wurde. \\ \\ \\

--------------------------------------- \\
Ort, Datum



\newpage

\newpage
{\huge \textbf{Danksagung}} \\ \\
An dieser Stelle m�chte ich mich bei all denjenigen bedanken, die mich w�hrend der Anfertigung dieser Master-Thesis unterst�tzt und motiviert haben. \\ \\
Zuerst geb�hrt mein Dank Herrn Prof. Dr.-Ing. Hellmut \textsc{Hupe} und Herrn J�rg \textsc{Fusenig}, der meine Master-Thesis richtungsweisend und mit viel Engagement betreut hat. F�r die hilfreichen Anregungen und die konstruktive Kritik bei der Erstellung dieser Arbeit m�chte ich mich herzlich bedanken.\\ 

%Des weiteren m�chte ich mich bei meinen Kommilitonen Michael Hengels, Lorenz Dirksmeyer und Alexander Schloeder f�r die gute Zusammenarbeit w�hrend des Studiums, die interessanten Diskussionen und Ideen bedanken. \\ 

Insbesondere m�chte ich mich bei meinen Eltern Egon und C�cilia Adams bedanken, die mir mein Studium durch ihre finanzielle Unterst�tzung erm�glicht haben und stets ein offenes Ohr f�r meine Sorgen hatten. \\ \\ \\
Michael P. Adams, \\
Trier, \today



\newpage
\cleardoublepage\pdfbookmark{\contentsname}{toc}\tableofcontents
\newpage



%\addcontentsline{toc}{chapter}{Abk�rzungsverzeichnis}
%\chapter*{Abk�rzungsverzeichnis}
%\begin{acronym}[DZGL]
%\acro{CW}[CW]{Continuous Wave}
%\end{acronym}




%\newpage

\addcontentsline{toc}{chapter}{Symbolverzeichnis}
\chapter*{Symbolverzeichnis}

\section*{Mathematische Symbole und Operatoren}

\begin{table}[H]
\begin{center}
\begin{tabular}{ c | c}
\parbox[c][0.8cm][c]{3cm}{\centering\textbf{Symbol} } 
& 
\parbox[c][0.8cm][c]{11cm}{\centering\textbf{Bezeichnung}}
\\
\hline %%%%%%%%%%%%%%%%%%%%%%%%%%%%%%%%%%%%%%%%%%%
\hline %%%%%%%%%%%%%%%%%%%%%%%%%%%%%%%%%%%%%%%%%%%
\parbox[c][0.8cm][c]{3cm}{\centering  $ \nabla $  }    
& 
\parbox[c][0.8cm][c]{11cm}{\centering  Nabla-Operator}
\\ 
\hline %%%%%%%%%%%%%%%%%%%%%%%%%%%%%%%%%%%%%%%%%%%
\parbox[c][0.8cm][c]{3cm}{\centering  $ \Delta $  }  
& 
\parbox[c][0.8cm][c]{11cm}{\centering  Laplace-Operator}
\\ 
\hline %%%%%%%%%%%%%%%%%%%%%%%%%%%%%%%%%%%%%%%%%%%
\parbox[c][0.8cm][c]{3cm}{\centering  $\mathbf{J}_{\underline{u}}\left( \underline{F}\right) $  }  
& 
\parbox[c][0.8cm][c]{11cm}{\centering  Jacobi-Matrix angewandt auf $\underline{F}$ bez�glich $\underline{u}$}
\\ 
\hline %%%%%%%%%%%%%%%%%%%%%%%%%%%%%%%%%%%%%%%%%%%
\parbox[c][0.8cm][c]{3cm}{\centering  $\mathcal{O}(\cdot) $  }  
& 
\parbox[c][0.8cm][c]{11cm}{\centering  Landau-Symbol}
\\ 
\hline %%%%%%%%%%%%%%%%%%%%%%%%%%%%%%%%%%%%%%%%%%%
\parbox[c][0.8cm][c]{3cm}{\centering  $\mathbf{0}$  }  
& 
\parbox[c][0.8cm][c]{11cm}{\centering  Nullmatrix}
\\
\hline %%%%%%%%%%%%%%%%%%%%%%%%%%%%%%%%%%%%%%%%%%%
\parbox[c][0.8cm][c]{3cm}{\centering  $\text{diag}(a_{11}, \cdots, a_{nn})$  }    
& 
\parbox[c][0.8cm][c]{11cm}{\centering  Diagonalmatrix}
\\
\hline %%%%%%%%%%%%%%%%%%%%%%%%%%%%%%%%%%%%%%%%%%%
\parbox[c][0.8cm][c]{3cm}{\centering  $\delta(\cdot)$  }  
& 
\parbox[c][0.8cm][c]{11cm}{\centering  Delta-Distribution}
\\
\hline %%%%%%%%%%%%%%%%%%%%%%%%%%%%%%%%%%%%%%%%%%%
\parbox[c][0.8cm][c]{3cm}{\centering  $\times$  }    
& 
\parbox[c][0.8cm][c]{11cm}{\centering  Kreuzprodukt}
\\
\hline %%%%%%%%%%%%%%%%%%%%%%%%%%%%%%%%%%%%%%%%%%%
\parbox[c][0.8cm][c]{3cm}{\centering  $\otimes$  }    
& 
\parbox[c][0.8cm][c]{11cm}{\centering  Dyadisches Produkt}
\\
\hline %%%%%%%%%%%%%%%%%%%%%%%%%%%%%%%%%%%%%%%%%%%
\parbox[c][0.8cm][c]{3cm}{\centering  $\overrightarrow{a}$  }  
& 
\parbox[c][0.8cm][c]{11cm}{\centering  Physikalischer 3D-Vektor}
\\
\hline %%%%%%%%%%%%%%%%%%%%%%%%%%%%%%%%%%%%%%%%%%%
\parbox[c][0.8cm][c]{3cm}{\centering  $\underline{a}$  }  
& 
\parbox[c][0.8cm][c]{11cm}{\centering  Generalisierter Vektor}
\\
\hline %%%%%%%%%%%%%%%%%%%%%%%%%%%%%%%%%%%%%%%%%%%
\parbox[c][0.8cm][c]{3cm}{\centering  $\text{E}\left\{\cdot\right\}$  }  
& 
\parbox[c][0.8cm][c]{11cm}{\centering  Erwartungswert-Operator}
\\
\hline %%%%%%%%%%%%%%%%%%%%%%%%%%%%%%%%%%%%%%%%%%%
\parbox[c][0.8cm][c]{3cm}{\centering  $\text{VAR}\left\{\cdot\right\}$  }  
& 
\parbox[c][0.8cm][c]{11cm}{\centering Varianz-Operator}
\\
\hline %%%%%%%%%%%%%%%%%%%%%%%%%%%%%%%%%%%%%%%%%%%
\parbox[c][0.8cm][c]{3cm}{\centering  $\mathbf{\Sigma}$  }  
& 
\parbox[c][0.8cm][c]{11cm}{\centering Kovarianz-Matrix}
\\
\hline %%%%%%%%%%%%%%%%%%%%%%%%%%%%%%%%%%%%%%%%%%%
\parbox[c][0.8cm][c]{3cm}{\centering  $\Lambda_{jj}$  }  
& 
\parbox[c][0.8cm][c]{11cm}{\centering Eigenwerte der Kovarianz-Matrix}
\\
\hline %%%%%%%%%%%%%%%%%%%%%%%%%%%%%%%%%%%%%%%%%%%
\parbox[c][0.8cm][c]{3cm}{\centering  $\mathbf{\Lambda}$  }  
& 
\parbox[c][0.8cm][c]{11cm}{\centering Diagonalisierte Kovarianz-Matrix}
\\
\hline %%%%%%%%%%%%%%%%%%%%%%%%%%%%%%%%%%%%%%%%%%%
\parbox[c][0.8cm][c]{3cm}{\centering  $\Upsilon_{ij}$  }  
& 
\parbox[c][0.8cm][c]{11cm}{\centering Komponenten der Eigenvektor-Matrix der Kovarianz-Matrix}
\\
\hline %%%%%%%%%%%%%%%%%%%%%%%%%%%%%%%%%%%%%%%%%%%
\parbox[c][0.8cm][c]{3cm}{\centering  $\mathbf{\Upsilon}$  }  
& 
\parbox[c][0.8cm][c]{11cm}{\centering Eigenvektor-Matrix der Kovarianz-Matrix}
\\
\end{tabular}
\end{center}
\end{table}%

\section*{Physikalische Konstanten}

\begin{table}[H]
\begin{center}
\begin{tabular}{ c | c}
\parbox[c][0.8cm][c]{5cm}{\centering\textbf{Symbol \& Wert} }  
& 
\parbox[c][0.8cm][c]{9cm}{\centering\textbf{Bezeichnung}}
\\
\hline %%%%%%%%%%%%%%%%%%%%%%%%%%%%%%%%%%%%%%%%%%%
\hline %%%%%%%%%%%%%%%%%%%%%%%%%%%%%%%%%%%%%%%%%%%
\parbox[c][0.8cm][c]{5cm}{\centering  $ \mu_0 = 4 \pi \cdot 10^{-7} \frac{\text{Vs}}{\text{Am}}  $  }    
& 
\parbox[c][0.8cm][c]{9cm}{\centering  Magnetische Feldkonstante}
\\ 
\hline %%%%%%%%%%%%%%%%%%%%%%%%%%%%%%%%%%%%%%%%%%%
\parbox[c][0.8cm][c]{5cm}{\centering  $ k_B = 1.380649 \cdot 10^{-23} \text{J}/\text{K}$  }   
& 
\parbox[c][0.8cm][c]{9cm}{\centering  Boltzmann-Konstante}
\\ 
\hline %%%%%%%%%%%%%%%%%%%%%%%%%%%%%%%%%%%%%%%%%%%
\parbox[c][0.8cm][c]{5cm}{\centering  $A \approx 10^{-19} \text{J}$  }  
& 
\parbox[c][0.8cm][c]{9cm}{\centering  Hamaker-Konstante}
\\ 
\end{tabular}
\end{center}
\end{table}%


\section*{Physikalische Gr��en}

\begin{table}[H]
\begin{center}
\begin{tabular}{ c | c | c}
\parbox[c][0.8cm][c]{2cm}{\centering\textbf{Symbol} }
&
\parbox[c][0.8cm][c]{2cm}{\centering \textbf{Einheit }}    
& 
\parbox[c][0.8cm][c]{8cm}{\centering\textbf{Bezeichnung}}
\\
\hline %%%%%%%%%%%%%%%%%%%%%%%%%%%%%%%%%%%%%%%%%%%
\hline %%%%%%%%%%%%%%%%%%%%%%%%%%%%%%%%%%%%%%%%%%%
\parbox[c][0.8cm][c]{2cm}{\centering  $ \psi_H $  }  
& 
\parbox[c][0.8cm][c]{2cm}{\centering  $\text{A}\cdot \text{m}^2$  }   
& 
\parbox[c][0.8cm][c]{10cm}{\centering  Skalarpotential der magnetischen Feldst�rke}
\\ 
\hline %%%%%%%%%%%%%%%%%%%%%%%%%%%%%%%%%%%%%%%%%%%
\parbox[c][0.8cm][c]{2cm}{\centering  $ \psi_B $  }  
& 
\parbox[c][0.8cm][c]{2cm}{\centering  $\text{T} \cdot \text{m}$  }   
& 
\parbox[c][0.8cm][c]{10cm}{\centering  Skalarpotential der magnetischen Flussdichte}
\\ 
\hline %%%%%%%%%%%%%%%%%%%%%%%%%%%%%%%%%%%%%%%%%%%
\parbox[c][0.8cm][c]{2cm}{\centering  $ \overrightarrow{H} $  }  
& 
\parbox[c][0.8cm][c]{2cm}{\centering  $\text{A}\cdot \text{m}$  }   
& 
\parbox[c][0.8cm][c]{10cm}{\centering  Magnetische Feldst�rke}
\\ 
\hline %%%%%%%%%%%%%%%%%%%%%%%%%%%%%%%%%%%%%%%%%%%
\parbox[c][0.8cm][c]{2cm}{\centering  $ \overrightarrow{B} $  }  
& 
\parbox[c][0.8cm][c]{2cm}{\centering  \text{T}  }   
& 
\parbox[c][0.8cm][c]{10cm}{\centering  Magnetische Flussdichte}
\\ 
\hline %%%%%%%%%%%%%%%%%%%%%%%%%%%%%%%%%%%%%%%%%%%
\parbox[c][0.8cm][c]{2cm}{\centering  $ \overrightarrow{m} $  }  
& 
\parbox[c][0.8cm][c]{2cm}{\centering  $\text{A} \cdot \text{m}^2$  }   
& 
\parbox[c][0.8cm][c]{10cm}{\centering  Magnetisches Moment}
\\ 
\hline %%%%%%%%%%%%%%%%%%%%%%%%%%%%%%%%%%%%%%%%%%%
\parbox[c][0.8cm][c]{2cm}{\centering  $ \overrightarrow{M} $  }  
& 
\parbox[c][0.8cm][c]{2cm}{\centering  $\text{A}/\text{m}$  }   
& 
\parbox[c][0.8cm][c]{10cm}{\centering  Magnetisierung}
\\ 
\hline %%%%%%%%%%%%%%%%%%%%%%%%%%%%%%%%%%%%%%%%%%%
\parbox[c][0.8cm][c]{2cm}{\centering  $ M $  }  
& 
\parbox[c][0.8cm][c]{2cm}{\centering  $\text{A}/\text{m}$  }   
& 
\parbox[c][0.8cm][c]{10cm}{\centering  Betrag der Magnetisierung}
\\ 
\hline %%%%%%%%%%%%%%%%%%%%%%%%%%%%%%%%%%%%%%%%%%%
\parbox[c][0.8cm][c]{2cm}{\centering  $ \Phi $  }  
& 
\parbox[c][0.8cm][c]{2cm}{\centering  $\text{Wb}$  }   
& 
\parbox[c][0.8cm][c]{10cm}{\centering  Magnetischer Remanenzfluss}
\\ 
\hline %%%%%%%%%%%%%%%%%%%%%%%%%%%%%%%%%%%%%%%%%%%
\parbox[c][0.8cm][c]{2cm}{\centering  $ \overrightarrow{l} $  }  
& 
\parbox[c][0.8cm][c]{2cm}{\centering  $\text{m}$  }   
& 
\parbox[c][0.8cm][c]{10cm}{\centering  Dipol-Achsenvektor}
\\ 
\hline %%%%%%%%%%%%%%%%%%%%%%%%%%%%%%%%%%%%%%%%%%%
\parbox[c][0.8cm][c]{2cm}{\centering  $ \overrightarrow{e} $  }  
& 
\parbox[c][0.8cm][c]{2cm}{\centering  -  }   
& 
\parbox[c][0.8cm][c]{10cm}{\centering  Dipol-Achseneinheitsvektor}
\\ 
\hline %%%%%%%%%%%%%%%%%%%%%%%%%%%%%%%%%%%%%%%%%%%
\parbox[c][0.8cm][c]{2cm}{\centering  $ \overrightarrow{F} $  }  
& 
\parbox[c][0.8cm][c]{2cm}{\centering  \text{N}  }   
& 
\parbox[c][0.8cm][c]{10cm}{\centering  Kraft}
\\ 
\hline %%%%%%%%%%%%%%%%%%%%%%%%%%%%%%%%%%%%%%%%%%%
\parbox[c][0.8cm][c]{2cm}{\centering  $ \overrightarrow{T} $  }  
& 
\parbox[c][0.8cm][c]{2cm}{\centering  $\text{N}\cdot \text{m}$  }   
& 
\parbox[c][0.8cm][c]{10cm}{\centering  Drehmoment}
\\
\hline %%%%%%%%%%%%%%%%%%%%%%%%%%%%%%%%%%%%%%%%%%%
\parbox[c][0.8cm][c]{2cm}{\centering  $ \overrightarrow{F_{ij}}^{D} $  }  
& 
\parbox[c][0.8cm][c]{2cm}{\centering  $\text{N}$  }   
& 
\parbox[c][0.8cm][c]{10cm}{\centering  Magnetische Dipol-Dipol-Kraft}
\\
\hline %%%%%%%%%%%%%%%%%%%%%%%%%%%%%%%%%%%%%%%%%%%
\parbox[c][0.8cm][c]{2cm}{\centering  $ \overrightarrow{T_{ij}}^{D} $  }  
& 
\parbox[c][0.8cm][c]{2cm}{\centering  $\text{N}\cdot \text{m}$  }   
& 
\parbox[c][0.8cm][c]{10cm}{\centering  Magnetisches Dipol-Dipol-Drehmoment}
\\
\hline %%%%%%%%%%%%%%%%%%%%%%%%%%%%%%%%%%%%%%%%%%%
\parbox[c][0.8cm][c]{2cm}{\centering  $ \overrightarrow{F_{ij}}^{H} $  }  
& 
\parbox[c][0.8cm][c]{2cm}{\centering  $\text{N}$  }   
& 
\parbox[c][0.8cm][c]{10cm}{\centering  Hamaker-Kraft}
\\
\hline %%%%%%%%%%%%%%%%%%%%%%%%%%%%%%%%%%%%%%%%%%%
\parbox[c][0.8cm][c]{2cm}{\centering  $ \overrightarrow{F_{ij}}^{S} $  }  
& 
\parbox[c][0.8cm][c]{2cm}{\centering  $\text{N}$  }   
& 
\parbox[c][0.8cm][c]{10cm}{\centering  Sterisch-repulsive Kraft}
\\
\hline %%%%%%%%%%%%%%%%%%%%%%%%%%%%%%%%%%%%%%%%%%%
\parbox[c][0.8cm][c]{2cm}{\centering  $ V_{ij}^{H} $  }  
& 
\parbox[c][0.8cm][c]{2cm}{\centering  $\text{J}$  }   
& 
\parbox[c][0.8cm][c]{10cm}{\centering  Hamaker-Potential }
\\
\hline %%%%%%%%%%%%%%%%%%%%%%%%%%%%%%%%%%%%%%%%%%%
\parbox[c][0.8cm][c]{2cm}{\centering  $ V_{ij}^{S} $  }  
& 
\parbox[c][0.8cm][c]{2cm}{\centering  $\text{J}$  }   
& 
\parbox[c][0.8cm][c]{10cm}{\centering  Potentielle Energie der sterischen Repulsion }
\\
\end{tabular}
\end{center}
\end{table}%


\begin{table}[H]
\begin{center}
\begin{tabular}{ c | c | c}
\parbox[c][0.8cm][c]{2cm}{\centering\textbf{Symbol} }
&
\parbox[c][0.8cm][c]{2cm}{\centering \textbf{Einheit }}    
& 
\parbox[c][0.8cm][c]{10cm}{\centering\textbf{Bezeichnung}}
\\
\hline %%%%%%%%%%%%%%%%%%%%%%%%%%%%%%%%%%%%%%%%%%%
\hline %%%%%%%%%%%%%%%%%%%%%%%%%%%%%%%%%%%%%%%%%%%
\parbox[c][0.8cm][c]{2cm}{\centering  $ \underline{q} $  }  
& 
\parbox[c][0.8cm][c]{2cm}{\centering  $-$}%$\begin{bmatrix} \text{m} & \text{m} & \text{m} & \text{rad} & \text{rad} \end{bmatrix}$  }   
& 
\parbox[c][0.8cm][c]{10cm}{\centering  Ortsvektor der generalisierten Koordinaten}
\\
\hline %%%%%%%%%%%%%%%%%%%%%%%%%%%%%%%%%%%%%%%%%%%
\parbox[c][0.8cm][c]{2cm}{\centering  $ \dot{\underline{q}} $  }  
& 
\parbox[c][0.8cm][c]{2cm}{\centering  $-$}%$\begin{bmatrix} \text{m} & \text{m} & \text{m} & \text{rad} & \text{rad} \end{bmatrix}$  }   
& 
\parbox[c][0.8cm][c]{10cm}{\centering  Geschwindigkeitsvektor der generalisierten Koordinaten}
\\
\hline %%%%%%%%%%%%%%%%%%%%%%%%%%%%%%%%%%%%%%%%%%%
\parbox[c][0.8cm][c]{2cm}{\centering  $ \ddot{\underline{q}} $  }  
& 
\parbox[c][0.8cm][c]{2cm}{\centering  $-$}%$\begin{bmatrix} \text{m} & \text{m} & \text{m} & \text{rad} & \text{rad} \end{bmatrix}$  }   
& 
\parbox[c][0.8cm][c]{10cm}{\centering  Beschleunigungsvektor der generalisierten Koordinaten}
\\
\hline %%%%%%%%%%%%%%%%%%%%%%%%%%%%%%%%%%%%%%%%%%%
\parbox[c][0.8cm][c]{2cm}{\centering  $ \lambda_1, \lambda_2, \lambda_3 $  }  
& 
\parbox[c][0.8cm][c]{2cm}{\centering  $-$}%$\begin{bmatrix} \text{m} & \text{m} & \text{m} & \text{rad} & \text{rad} \end{bmatrix}$  }   
& 
\parbox[c][0.8cm][c]{10cm}{\centering  K�rperfeste Kugelkoordinaten}
\\
\hline %%%%%%%%%%%%%%%%%%%%%%%%%%%%%%%%%%%%%%%%%%%
\parbox[c][0.8cm][c]{2cm}{\centering  $ \mathbfcal{J} $  }  
& 
\parbox[c][0.8cm][c]{2cm}{\centering  -  }   
& 
\parbox[c][0.8cm][c]{10cm}{\centering  \textsc{Jacobi}-Matrix (\textsc{d'Alembert}'sches Prinzip)}
\\
\hline %%%%%%%%%%%%%%%%%%%%%%%%%%%%%%%%%%%%%%%%%%%
\parbox[c][0.8cm][c]{2cm}{\centering  $ \mathbfcal{M} $  }  
& 
\parbox[c][0.8cm][c]{2cm}{\centering  -  }   
& 
\parbox[c][0.8cm][c]{10cm}{\centering  Massenmatrix (\textsc{d'Alembert}'sches Prinzip)}
\\
\hline %%%%%%%%%%%%%%%%%%%%%%%%%%%%%%%%%%%%%%%%%%%
\parbox[c][0.8cm][c]{2cm}{\centering  $ \mathbfcal{R} $  }  
& 
\parbox[c][0.8cm][c]{2cm}{\centering  -  }   
& 
\parbox[c][0.8cm][c]{10cm}{\centering  Relativmatrix (\textsc{d'Alembert}'sches Prinzip)}
\\
\hline %%%%%%%%%%%%%%%%%%%%%%%%%%%%%%%%%%%%%%%%%%%
\parbox[c][0.8cm][c]{2cm}{\centering  $ \underline{F} $  }  
& 
\parbox[c][0.8cm][c]{2cm}{\centering  -  }   
& 
\parbox[c][0.8cm][c]{10cm}{\centering  Generalisierter Kraftvektor (\textsc{d'Alembert}'sches Prinzip)}
\\
\hline %%%%%%%%%%%%%%%%%%%%%%%%%%%%%%%%%%%%%%%%%%%
\parbox[c][0.8cm][c]{2cm}{\centering  $ x, y, z $  }  
& 
\parbox[c][0.8cm][c]{2cm}{\centering $ \text{m}$  }   
& 
\parbox[c][0.8cm][c]{10cm}{\centering  Kartesische Koorindaten}
\\ 
\hline %%%%%%%%%%%%%%%%%%%%%%%%%%%%%%%%%%%%%%%%%%%
\parbox[c][0.8cm][c]{2cm}{\centering  $ \alpha $  }  
& 
\parbox[c][0.8cm][c]{2cm}{\centering $ \text{rad}$  }   
& 
\parbox[c][0.8cm][c]{10cm}{\centering  Azimut-Winkel}
\\ 
\hline %%%%%%%%%%%%%%%%%%%%%%%%%%%%%%%%%%%%%%%%%%%
\parbox[c][0.8cm][c]{2cm}{\centering  $ \beta $  }  
& 
\parbox[c][0.8cm][c]{2cm}{\centering $ \text{rad}$  }   
& 
\parbox[c][0.8cm][c]{10cm}{\centering  Zenit-Winkel}
\\ 
\hline %%%%%%%%%%%%%%%%%%%%%%%%%%%%%%%%%%%%%%%%%%%
\parbox[c][0.8cm][c]{2cm}{\centering  $ \overrightarrow{r_{ij}} $  }  
& 
\parbox[c][0.8cm][c]{2cm}{\centering $ \text{m}$  }   
& 
\parbox[c][0.8cm][c]{10cm}{\centering  Abstandsvektor von $i$ zu $j$}
\\ 
\hline %%%%%%%%%%%%%%%%%%%%%%%%%%%%%%%%%%%%%%%%%%%
\parbox[c][0.8cm][c]{2cm}{\centering  $\eta $  }  
& 
\parbox[c][0.8cm][c]{2cm}{\centering $  \text{kg}/(\text{m} \cdot \text{s}) $  }   
& 
\parbox[c][0.8cm][c]{10cm}{\centering  Dynamische Viskosit�t}
\\ 
\hline %%%%%%%%%%%%%%%%%%%%%%%%%%%%%%%%%%%%%%%%%%%
\parbox[c][0.8cm][c]{2cm}{\centering  $T$  }  
& 
\parbox[c][0.8cm][c]{2cm}{\centering $\text{K}$  }   
& 
\parbox[c][0.8cm][c]{10cm}{\centering  Temperatur}
\\ 
\hline %%%%%%%%%%%%%%%%%%%%%%%%%%%%%%%%%%%%%%%%%%%
\parbox[c][0.8cm][c]{2cm}{\centering  $R$  }  
& 
\parbox[c][0.8cm][c]{2cm}{\centering $\text{m}$  }   
& 
\parbox[c][0.8cm][c]{10cm}{\centering  Partikel-Radius}
\\ 
\hline %%%%%%%%%%%%%%%%%%%%%%%%%%%%%%%%%%%%%%%%%%%
\parbox[c][0.8cm][c]{2cm}{\centering  $\rho$  }  
& 
\parbox[c][0.8cm][c]{2cm}{\centering $\text{kg}/\text{m}^3$  }   
& 
\parbox[c][0.8cm][c]{10cm}{\centering Massendichte}
\\
\hline %%%%%%%%%%%%%%%%%%%%%%%%%%%%%%%%%%%%%%%%%%%
\parbox[c][0.8cm][c]{2cm}{\centering  $m$  }  
& 
\parbox[c][0.8cm][c]{2cm}{\centering $\text{kg}$  }   
& 
\parbox[c][0.8cm][c]{10cm}{\centering Masse}
\\
\hline %%%%%%%%%%%%%%%%%%%%%%%%%%%%%%%%%%%%%%%%%%%
\parbox[c][0.8cm][c]{2cm}{\centering  $V$  }  
& 
\parbox[c][0.8cm][c]{2cm}{\centering $\text{m}^3$  }   
& 
\parbox[c][0.8cm][c]{10cm}{\centering Partikel-Volumen}
\\
\hline %%%%%%%%%%%%%%%%%%%%%%%%%%%%%%%%%%%%%%%%%%%
\parbox[c][0.8cm][c]{2cm}{\centering  $\gamma_t$  }  
& 
\parbox[c][0.8cm][c]{2cm}{\centering $\text{kg}/\text{s}$  }   
& 
\parbox[c][0.8cm][c]{10cm}{\centering Transversale Stokes'sche Reibungskonstante }
\\
\hline %%%%%%%%%%%%%%%%%%%%%%%%%%%%%%%%%%%%%%%%%%%
\parbox[c][0.8cm][c]{2cm}{\centering  $\gamma_r$  }  
& 
\parbox[c][0.8cm][c]{2cm}{\centering $(\text{kg} \cdot \text{m}^2)/\text{s}$  }   
& 
\parbox[c][0.8cm][c]{10cm}{\centering Rotatorische Stokes'sche Reibungskonstante }
\\
\hline %%%%%%%%%%%%%%%%%%%%%%%%%%%%%%%%%%%%%%%%%%%
\parbox[c][0.8cm][c]{2cm}{\centering  $\xi$  }  
& 
\parbox[c][0.8cm][c]{2cm}{\centering $1/\text{m}^2$  }   
& 
\parbox[c][0.8cm][c]{10cm}{\centering Dichte der Tensidschicht}
\\
\hline %%%%%%%%%%%%%%%%%%%%%%%%%%%%%%%%%%%%%%%%%%%
\parbox[c][0.8cm][c]{2cm}{\centering  $\delta$  }  
& 
\parbox[c][0.8cm][c]{2cm}{\centering $\text{m}$  }   
& 
\parbox[c][0.8cm][c]{10cm}{\centering Dicke der Tensidschicht }
\\
\end{tabular}
\end{center}
\end{table}%

\newpage

\section*{Prim�re Skalierungs-Konstanten}

\begin{table}[H]
\begin{center}
\begin{tabular}{ c | c | c}
\parbox[c][0.8cm][c]{2cm}{\centering\textbf{Symbol} }
&
\parbox[c][0.8cm][c]{2cm}{\centering \textbf{Einheit }}    
& 
\parbox[c][0.8cm][c]{10cm}{\centering\textbf{Bezeichnung}}
\\
\hline %%%%%%%%%%%%%%%%%%%%%%%%%%%%%%%%%%%%%%%%%%%
\hline %%%%%%%%%%%%%%%%%%%%%%%%%%%%%%%%%%%%%%%%%%%
\parbox[c][0.8cm][c]{2cm}{\centering  $T_{\text{ch}} $  }  
& 
\parbox[c][0.8cm][c]{2cm}{\centering  $\text{s}$  }   
& 
\parbox[c][0.8cm][c]{10cm}{\centering  Zeitlicher Skalierungsfaktor}
\\ 
\hline %%%%%%%%%%%%%%%%%%%%%%%%%%%%%%%%%%%%%%%%%%%
\parbox[c][0.8cm][c]{2cm}{\centering  $R $  }  
& 
\parbox[c][0.8cm][c]{2cm}{\centering  $\text{m}$  }   
& 
\parbox[c][0.8cm][c]{10cm}{\centering  R�umlicher Skalierungsfaktor}
\\ 
\end{tabular}
\end{center}
\end{table}%

\section*{Skalierte Kofaktoren der Kr�fte}

\begin{table}[H]
\begin{center}
\begin{tabular}{ c | c | c}
\parbox[c][0.8cm][c]{2cm}{\centering\textbf{Symbol} }
&
\parbox[c][0.8cm][c]{2cm}{\centering \textbf{Einheit }}    
& 
\parbox[c][0.8cm][c]{10cm}{\centering\textbf{Bezeichnung}}
\\
\hline %%%%%%%%%%%%%%%%%%%%%%%%%%%%%%%%%%%%%%%%%%%
\hline %%%%%%%%%%%%%%%%%%%%%%%%%%%%%%%%%%%%%%%%%%%
\parbox[c][0.8cm][c]{2cm}{\centering  $C_H$  }  
& 
\parbox[c][0.8cm][c]{2cm}{\centering  $-$  }   
& 
\parbox[c][0.8cm][c]{10cm}{\centering  Kofaktor der Hamaker-Kraft}
\\ 
\hline %%%%%%%%%%%%%%%%%%%%%%%%%%%%%%%%%%%%%%%%%%%
\parbox[c][0.8cm][c]{2cm}{\centering  $C_S $  }  
& 
\parbox[c][0.8cm][c]{2cm}{\centering  $-$  }   
& 
\parbox[c][0.8cm][c]{10cm}{\centering  Kofaktor 1 der sterischen Repulsion}
\\ 
\hline %%%%%%%%%%%%%%%%%%%%%%%%%%%%%%%%%%%%%%%%%%%
\parbox[c][0.8cm][c]{2cm}{\centering  $c_S $  }  
& 
\parbox[c][0.8cm][c]{2cm}{\centering  $-$  }   
& 
\parbox[c][0.8cm][c]{10cm}{\centering  Kofaktor 2 der sterischen Repulsion}
\\ 
\hline %%%%%%%%%%%%%%%%%%%%%%%%%%%%%%%%%%%%%%%%%%%
\parbox[c][0.8cm][c]{2cm}{\centering  $C_{FD} $  }  
& 
\parbox[c][0.8cm][c]{2cm}{\centering  $-$  }   
& 
\parbox[c][0.8cm][c]{10cm}{\centering  Kofaktor der magn. Dipol-Dipol-Kraft}
\\ 
\hline %%%%%%%%%%%%%%%%%%%%%%%%%%%%%%%%%%%%%%%%%%%
\parbox[c][0.8cm][c]{2cm}{\centering  $C_{TD} $  }  
& 
\parbox[c][0.8cm][c]{2cm}{\centering  $-$  }   
& 
\parbox[c][0.8cm][c]{10cm}{\centering  Kofaktor des magn. Dipol-Dipol-Drehmoment}
\\ 
\hline %%%%%%%%%%%%%%%%%%%%%%%%%%%%%%%%%%%%%%%%%%%
\parbox[c][0.8cm][c]{2cm}{\centering  $C_{FB} $  }  
& 
\parbox[c][0.8cm][c]{2cm}{\centering  $\frac{\text{s}^2\cdot \text{m} \cdot \text{A}}{\text{kg}}$  }   
& 
\parbox[c][0.8cm][c]{10cm}{\centering  Kofaktor �u�erer magn. Kr�fte}
\\ 
\hline %%%%%%%%%%%%%%%%%%%%%%%%%%%%%%%%%%%%%%%%%%%
\parbox[c][0.8cm][c]{2cm}{\centering  $C_{TB} $  }  
& 
\parbox[c][0.8cm][c]{2cm}{\centering  $\frac{\text{s}^2  \cdot \text{A}}{\text{kg}}$  }   
& 
\parbox[c][0.8cm][c]{10cm}{\centering  Kofaktor �u�erer magn. Drehmomente}
\\ 
\end{tabular}
\end{center}
\end{table}%

 

% Abbildungs- und Tabellenverzeichnis %%%%%%%%%%%%%%%%%%%%%%%%%%%%%%%%%%%%%%%%%%%%%%%%
\newpage
\noindent
%\begin{minipage}{\textwidth}
\listoffigures
\addcontentsline{toc}{chapter}{Abbildungsverzeichnis}
\listoftables
\addcontentsline{toc}{chapter}{Tabellenverzeichnis}
%\end{minipage}

\chapter{Title}

\section{Title}









% Literaturverzeichnis %%%%%%%%%%%%%%%%%%%%%%%%%%%%%%%%%%%%%%%%%%%%%%%%
\newpage
%%%%%\medskip
%%%%\printbibliography

\bibliography{references}
\addcontentsline{toc}{chapter}{Literaturverzeichnis}
%\nocite{*}
%%%%%%%%%%%%%%%%%%%%%%%%%%%%%%%%%%%%%%%%%%%%%%%%%%%%%%%%%%%


\end{document}

